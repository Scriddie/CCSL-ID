\documentclass{article}

\usepackage[left=1in, right=1in, top=1in, bottom=1in]{geometry}
\usepackage{hyperref}
\usepackage[at]{easylist}
\usepackage[
    doi=false,isbn=false,url=false,
    backend=biber,
    uniquename=false,
    bibencoding=utf8,
    sorting=nty,
    citestyle=authoryear,
    ]{biblatex}
    \addbibresource{bibliography.bib}

\title{Reading Notes}
\author{Alexander Reisach}

\begin{document}
    
\maketitle

\section{Plan}
\begin{easylist}
    \ListProperties(Style1*=\bfseries,Numbers2=l,Mark1={},Mark2={)},Indent2=1em)
    @ Start by looking at data generation methods
    @@ Can I find the \cite{sachs2005causal} pathways in \cite{belinky2015pathcards} and \cite{perfetto2016signor}?
    @@ Do \cite{van2006syntren} and \cite{schaffter2011genenetweaver} produce varsortable data?
    @ Take a look at real-world data
    @ Implement \cite{brouillard2020differentiable}
    @ Consider transfer learning regret
    @ Try cycle breaking
\end{easylist}

\section{Related Literature}
\cite{itani2010structure} provide a good recipe for learning cyclical causal structure from interventional data.
\cite{vowels2021} give a comprehensive overview over all kinds of structure learning methods and benchmarks.
\cite{van2006syntren} provide a good overview over other network simulations.

\subsection{Datasets}

I am not sure there are much better data sets than the one by \cite{sachs2005causal}.

\paragraph{SynTReN}
is a synthetic TRN data generation scheme proposed by \cite{van2006syntren}. 
They talk about ODE simulations. Should have a look at them.
Check out the original source networks they consider, also look for more up-to-date networks
They seem to just do additive noise models
Visualize and compare that to Sachs et al and another dataset from vowels2021

\paragraph{GeneNetWeaver} is a synthetic TRN data generation scheme proposed by \cite{schaffter2011genenetweaver}. Their model was used in benchmarking the Dialogue for Reverse Engineering Assessments and Methods (DREAM) challenges.

\paragraph{SIGNOR} is a database of established causal relationships between biological entities \cite{perfetto2016signor}.

\paragraph{PathCards} unifies several sources on human biological pathways \cite{belinky2015pathcards}.
They provide a good overview over other pathway sources.

\subsection{Benchmarking}

Ernest Fraenkel\footnote{\url{https://www.youtube.com/watch?v=RBPcKbEvK3U&list=PLUl4u3cNGP63uK-oWiLgO7LLJV6ZCWXac&index=17}} gives a very nice overview over \cite{marbach2012wisdom} where he explains how some funamental assumptions may be violated in real-world cell signalling data.
He shows that unlike in in-silico data and E.coli data, for yeast co-regulated genes are basically uncorrelated and have no mutual information.

Tuebingen causality pairs \url{https://webdav.tuebingen.mpg.de/cause-effect/} (only observational).
Try varsortability in the bivariate case?

\section{Conceptional Work}
    - Leap of abstraction from PDE to causal graph

\section{Methodology}
\begin{itemize}
    \item Only apply acyclicity constraint gradually/at a later stage?
    \item Do not optimize over diagonal?
    \item Use dropout (predictions/gradients)?
    \item Combine continuous structure learning with cyclic relationships
\end{itemize}

\subsection{Transfer Learning Regret}
Do \cite{bengio2019meta} normalize? if not, isn't maybe the variance in one direction just bigger than in the other which requires more steps? In the high-variance direction, we might see bigger losses at the start $\dots$
How do they generate their data?

In Bengio's paper, are we ultimately just talking about inverting a (mathematically) irreversible function?
Isn't that conceptionally exactly the same as the light switch example where X inevitably causes Y but Y can also be caused by many other things?

Is there something to be gained from viewing data as a sequence in the style of \cite{bengio2019meta}? Mybe try transfer learning regret on bio data?

Is it maybe enough to give the conditional more params to fit than the marginal? :O
Actually, can we just test which way the marginal changes more? The conditional will have to account for the rest of the variation...

Is there a way to break causality by including the wrong things? Not just by treating them in the wrong way, but by including them in the question at all!

\printbibliography

\end{document}