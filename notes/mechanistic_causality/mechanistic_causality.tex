\documentclass{article}
\usepackage{amsmath}
\usepackage[left=1.2in, right=1.2in, top=1.2in, bottom=1.2in]{geometry}

\title{Mechanistic Causality}
\author{Alexander Reisach}

\begin{document}
\maketitle
    
\section{Introduction}
We define interventional dependence and directionality of the dependence as devining characteristics of a causal relationship between two variables.
For our purposes, we observe the equilibrium states $\epsilon_t$ with $t \in 1 \dots n$ of a closed system $S$ with known components $c_i$ with $i \in 1 \dots n$. 

\subsection{Causal Derivatives}
Causal BMI example!

\subsection{Causality Between Dimensionally Independent SI units}
We postulate that causal effects are impossible between variables which are measured in dimensionally dependent units. Between 

\paragraph{Example} In a chamber full of negatively charged particles, the magnitude of their charge could be conceived as the cause for their relative location/distance from one another. The relationship is not reciprocal due to the dimensional independence of the units. By contrast, the force between them would not be a cause of the location, as a change in location would also change the force between them. The two quantities share a SI dimension and the effect is reciprocal.

\subsection{Causality as Disproportionality}
Causality could be conceived of as disproportionality in a reciprocal relationship. 

\paragraph{Example} Imagine the same chamber of charged particles as before. I one of the particles had a mass vastly greater than any of the others, its location would have a much bigger impact on the location of the other particles than vice versa.\\

This effect is not only conceivable between two variables associated with real-world objects. It could also arise in a scenario where one variable measures an aggregate quantity of a collection of individual objects.

Accepting this definition of causality would imply that the data scale is indeed crucial for determining causal relationships

\end{document}